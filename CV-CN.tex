%%%%%%%%%%%%%%%%%%%%%%%%%%%%%%%%%%%%%
% Document properties and packages
%%%%%%%%%%%%%%%%%%%%%%%%%%%%%%%%%%%%%
\documentclass[a4paper,12pt,final,UTF8,fontset=macnew]{memoir}
\usepackage{ctex}%中文支持
\usepackage{newtxtext}

% misc
%\renewcommand{\familydefault}{bch}	% font
\pagestyle{empty}					% no pagenumbering
\setlength{\parindent}{0pt}			% no paragraph indentation
% required packages (add your own)
\usepackage{flowfram}% column layou
\usepackage{marvosym}
\usepackage{amsmath}
\usepackage{textcomp}
\usepackage{varwidth}
\usepackage{fontawesome}
\usepackage{color}

\usepackage[top=1cm,left=1cm,right=1cm,bottom=1cm]{geometry}% margins
\usepackage{graphicx}										% figures
\usepackage{hyperref}
\definecolor{linkcolour}{rgb}{0,0.2,0.6}  %蓝色
\hypersetup{colorlinks,breaklinks,urlcolor=linkcolour, linkcolor=linkcolour}										% URLs
\usepackage[usenames,dvipsnames]{xcolor}					% color
\usepackage{multicol}										% columns env.
	\setlength{\multicolsep}{0pt}
\usepackage{paralist}										% compact lists
\usepackage{tikz}
\usepackage{tikzpeople}
\usetikzlibrary{shapes.geometric,calc}
\usepackage{tcolorbox}
\usepackage{enumitem}
\newcommand\score[2]{%
	\pgfmathsetmacro\pgfxa{#1 + 1}%
	\tikzstyle{scorestars}=[star, star points=5, star point ratio=2.25, draw, inner sep=1.3pt, anchor=outer point 3]%
	\begin{tikzpicture}[baseline]
	\foreach \i in {1, ..., #2} {
		\pgfmathparse{\i<=#1 ? "yellow" : "gray"}
		\edef\starcolor{\pgfmathresult}
		\draw (\i*1.75ex, 0) node[name=star\i, scorestars, fill=\starcolor]  {};
	}
	\end{tikzpicture}%
}


\setlength{\itemsep}{0.1pt}
%\usetikzlibrary{shapes.geometric}
%%%%%%%%%%%%%%%%%%%%%%%%%%%%%%%%%%%%%
% Create column layout
%%%%%%%%%%%%%%%%%%%%%%%%%%%%%%%%%%%%%
% define length commands
\setlength{\vcolumnsep}{\baselineskip}
\setlength{\columnsep}{\vcolumnsep}
%定义主题颜色,可选颜色 Maroon,ForestGreen,DarkOrchid,RoyalBlue,Turquoise,Cyan,etc,更多颜色参考xcolor包的颜色定义
\newcommand{\myThemeColor}{RoyalBlue}
% frame setup (flowfram package)
% left frame
\newflowframe{0.23\textwidth}{\textheight}{0pt}{0pt}[left]
	\newlength{\LeftMainSep}
	\setlength{\LeftMainSep}{0.23\textwidth}
	\addtolength{\LeftMainSep}{1\columnsep}

% small static frame for the vertical line
\newstaticframe{1.5pt}{\textheight}{\LeftMainSep}{0pt}


% content of the static frame
\begin{staticcontents}{1} %绘制分割线,使用tikz包绘制。如需改变风格线样式,请参考tikz教程,对于新手,不建议修改。
\hfill
\tikz{%
	\draw[loosely dotted,color=\myThemeColor,line width=1.5pt,yshift=0]
	(0,0) -- (0,\textheight);}%
\hfill\mbox{}
\end{staticcontents}

% right frame
\addtolength{\LeftMainSep}{1.5pt}
\addtolength{\LeftMainSep}{1\columnsep}
\newflowframe{0.73\textwidth}{\textheight}{\LeftMainSep}{0pt}[main01]


%%%%%%%%%%%%%%%%%%%%%%%%%%%%%%%%%%%%%
% define macros (for convience)
%%%%%%%%%%%%%%%%%%%%%%%%%%%%%%%%%%%%%
\newcommand{\Sep}{\vspace{1em}}
\newcommand{\SmallSep}{\vspace{0.9em}}

\newenvironment{AboutMe}
	{\ignorespaces\textbf{\color{\myThemeColor} About me}}
	{\Sep\ignorespacesafterend}
%定义section	
\newcommand{\CVSection}[1]
	{\Large\textbf{#1}\par\smallskip
     \hrule%
     \smallskip
	 \normalsize\normalfont}

\newcommand{\CVItem}[1]
	{\textbf{\color{\myThemeColor} #1}}


%%%%%%%%%%%%%%%%%%%%%%%%%%%%%%%%%%%%%
% Begin document
%%%%%%%%%%%%%%%%%%%%%%%%%%%%%%%%%%%%%
\begin{document}
\linespread{1.15}\selectfont
% Left frame 左边内容在此定义
%%%%%%%%%%%%%%%%%%%%
\begin{figure}
\centering
	\includegraphics[width=0.8\columnwidth]{dog.jpg}
%\tikz\node[graduate,minimum height=4cm]{};	
\vspace{-7cm}
\end{figure}
\begin{flushright}\footnotesize
.\\
\vskip 6cm
    \raggedright
	\CVItem{{\large \faInfoCircle 个人信息:}}\\
	{\zihao{-4}
	\textcolor{blue}{\faEnvelope} 电子邮件:\\
	15958174339@163.com \\
	 \includegraphics[scale=0.07]{csdn.pdf} \hspace{-0.3em} CSDN:\\	
	 \href{https://blog.csdn.net/weixin_43342986}{\small https://blog.csdn.net/weixin\_ 43342986 } \\
	\faGithub  \ Github:\\
	\href{https://github.com/neverland-haha?tab=repositories}{\small https://github.com/neverland-haha?tab=repositories}   \\
	\textcolor{red}{\faPhone}手机:\\
	15958174339\\
	17861511558 \\ }
	\vspace{3em}
	\CVItem{{\vspace{-1.2em}\large \faLanguage 语言能力:}}\\
	
	{\zihao{-4}\textit{\\ 四级:616\\  六级:520\\
		    	四级口语:B\\
		    	六级口语:B+
			}  }
	
	\vspace{3em}
	\CVItem{{\large \faDesktop 电脑技能:}}\\[1em]
	\textbf{\zihao{4} \faKeyboardO 编程语言: }\\ 
	{\zihao{-4}
	$\bullet$ \textbf{MATLAB:}  	\score{3}{5}  \\
	$\bullet$ \textbf{Python:}  \score{2}{5} \\
	$\bullet$ \LaTeX:      \score{3}{5} \\
	$\bullet$ \textbf{IDL}   \score{2}{5} \\ }
	\vspace{4.3em}
	\CVItem{\large  \faLaptop 专业软件:} \\   {\zihao{-4}$\bullet $Envi \score{3}{5}  \\
		$\bullet$  ArcGis \score{2}{5}  \\
		$\bullet$	MRT	   \score{2}{5} \\  }
		\vspace{4.3em}
	\CVItem{{\large \faThumbsOUp 爱好:}}\\
	\textit{跑步、打羽毛球、编程、思考、美剧}
	\SmallSep
\end{flushright}\normalsize
\framebreak


% Right frame 右边内容在此定义
%%%%%%%%%%%%%%%%%%%%
%\Huge\bfseries {\color{\myThemeColor} 某~某~某}\\[-15pt]
\normalsize\normalfont

% Education
\CVSection{基本情况} 
	\begin{table}[h]
	\small
	\begin{tabular*}{\textwidth}{l@{\extracolsep{0.4em}}lll}
	姓名:Neverland!   &   性别:男        &出生年月:1997年8月       \\[1em]
	籍贯:乌有乡     &现居地:乌托邦       &民族:汉      \\[1em]
	政治面貌:共青团员    &就读院校:山东种田大学      &本科专业:遥感科学与技术  \\[1em]
	 英语水平:CET6 &三年成绩: 4.06/5               &专业排名:4/54
	\end{tabular*}
	\end{table}

\vspace{1em}
%self comment
\CVSection{自我评价及未来打算}
	 \vspace{1em}
	 {
	 \begin{varwidth}{32em}
	 	 \setlength\parindent{2em}
	  \par 性格上比较沉稳,对技术有着近乎狂热的追求,有上进心,有较强的实践动能力,接受新知识较为快,倘若无法接受,会花很长的时间去思考,直至明白原理之后再.英语水平较好,喜欢阅读源网站的documentation.善于总结,喜欢通过一些特殊的方式去记录所学的东西.同时擅长反思,喜欢在夜晚的时候跑步总结思考所学和所疑惑.
	  \par 对夜光遥感非常感兴趣,毕业设计打算利用DMSP和NPP-VIIRS结合人口迁数据况做长三角的经济发展状况的评价.同时对图像处理也较为感兴趣,研究生期间准备攻读图像处理方面的领域.
	  \end{varwidth}
	}
% Experience
\vspace{2em}

% CAMPU
\CVSection{综合素质及科研竞赛} 
	\begin{varwidth}{32em}
	\begin{enumerate}
		\item 擅长编程,能够熟练使用MATLAB,\LaTeX.
		\item 写有20余篇博客,包括MATLAB、\LaTeX、数据下载、数据处理等。
		\item 主持一项大学生创新创业计划,主要负责数据下载和数据处理.
		\item 参加大学生数学建模大赛,虽未获奖,但学习了如何阅读文献及、快速分析问题、科学写事项等方面的事项.
		\item 大三摄影测量课程的专业课程设计用MATLAB纯代码形式写Canny边缘提取的UI.
		\item 编写了山东农业大学信息学院的\LaTeX 模板。
	\end{enumerate}
	\end{varwidth}

% HONORS & SCHOLARSHIPS
\vspace{2em}
\CVSection{个人荣誉} 
\SmallSep
	\begin{table}[h]
	\begin{tabular}{l|l}
		$\Rightarrow$ 2018.05&\textit{大学生英语竞赛三等奖}\footnotesize\\
		$\Rightarrow$ 2018.09&\textit{山东农业大学优秀学生}\\
		$\Rightarrow$ 2018.09&\textit{山东农业大学二等奖学金}\\
		$\Rightarrow$ 2018.10&\textit{大学生数学竞赛省三等奖}\\
		$\Rightarrow$ 2018.10&\textit{外研杯山东农业大学英语阅读比赛一等奖}\\
		$\Rightarrow$ 2019.09&\textit{山东农业大学数学建模比赛二等奖}\\
		$\Rightarrow$ 2019.09&\textit{山东农业大学二等奖学金}\footnotesize\\ 
		$\Rightarrow$ 2020.07&\textit{编写了山东农业大学\LaTeX 模板,获数学系老师认证}\footnotesize\\
		$\Rightarrow$ 2020.07&\textit{参加北京师范大学定量遥感暑期学校,获得结业证书}\footnotesize\\
		
	\end{tabular}
\end{table}
%%%%%%%%%%%%%%%%%%%%%%%%%%%%%%%%%%%%%
% End document
%%%%%%%%%%%%%%%%%%%%%%%%%%%%%%%%%%%%%

\end{document} 